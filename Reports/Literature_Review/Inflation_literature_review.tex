\documentclass[11pt]{article}

%% Package Importing
%%-----------------------------------------------------------------------------------------------------------------------------
\usepackage{outline}
\usepackage{geometry}
\geometry{a4paper, margin=1in}
\usepackage{helvet}
\usepackage{graphicx}
\usepackage{physics}
\usepackage{titlesec}
\usepackage[backend=biber,style=numeric,sorting=none]{biblatex}
\addbibresource{Phys450-References.bib}
\usepackage{fancyhdr}
\usepackage{titling}
\usepackage[hidelinks]{hyperref}
\usepackage{cleveref}
\usepackage{multicol}
\usepackage{cases}
\usepackage{float}
\usepackage{booktabs}
\usepackage{multirow}
\usepackage{bookmark}
\usepackage{footmisc}
\usepackage{caption}
\usepackage{wrapfig}
\usepackage[inkscapepath=svg-inkscape/]{svg}
\captionsetup[figure]{font=footnotesize}
\captionsetup[table]{font=footnotesize}


%% Defines the page environments
%%-----------------------------------------------------------------------------------------------------------------------------
\fancypagestyle{titlepage}{
    \setlength{\headheight}{13.6pt}
    \fancyhead[L]{Lancaster University}
    \fancyhead[R]{Department of Physics}
    \fancyfoot[C]{\thepage}
	\renewcommand{\headrulewidth}{1pt}
	\renewcommand{\footrulewidth}{1pt} 
}

\fancypagestyle{subsequentpages}{
    \fancyfoot[C]{\thepage}
	\renewcommand{\headrulewidth}{1pt}
	\renewcommand{\footrulewidth}{1pt}
}

%% Configures spacing for equations
%%-----------------------------------------------------------------------------------------------------------------------------
\AtBeginDocument{
  \setlength{\abovedisplayskip}{2pt}
  \setlength{\belowdisplayskip}{2pt}
  \setlength{\abovedisplayshortskip}{2pt}
  \setlength{\belowdisplayshortskip}{2pt}
}

%% \renewcommand lines
%%-----------------------------------------------------------------------------------------------------------------------------
\renewcommand{\arraystretch}{1.5}
\sloppy
\linespread{0.9}
\crefname{equation}{Eq.}{Eqs.}
\crefname{figure}{Fig.}{Figs.}
\crefname{section}{Sec.}{Secs.}

%% Begin Document
%%-----------------------------------------------------------------------------------------------------------------------------
\begin{document}

%% Equation numbering
%%-----------------------------------------------------------------------------------------------------------------------------
\numberwithin{equation}{section}
\setcounter{equation}{0}
\numberwithin{equation}{subsection}
\setcounter{equation}{0}


%% Title Page
%%-----------------------------------------------------------------------------------------------------------------------------
\begin{titlepage}
    \thispagestyle{titlepage}
    \pagenumbering{roman}
    \centering
    \includesvg[width=0.7\linewidth]{Images/PhysicsLogo}\\
    \vspace{0.5cm}
    {\Huge \textbf{An overview of Cosmological Inflation} \par}
    \vspace{1cm}
    {\large Rhys Abercromby\par}
    \vspace{0.5cm}
    {\large Phys450: MPhys Project \par}
    \vspace{0.5cm}
    {\today \par}
    \vspace{1cm}
    {\large \textbf{Abstract} \par}
    \vspace{0.5cm}
    \sloppy
    Despite the successes of the Big Bang Theory, there are problems that occur when examining the very earliest moments of the universe. The Inflationary Paradigm is a potential answer to these problems; however, there are a plethora of different models that explain this in their own way. As of now, no model has been proven, with the most successful ones being Starobinsky Inflation and $\alpha$-attractor models. These models are those that best align with the Planck 2018 observational bounds of the spectral index and tensor ratio, though there is potential for observations to disprove the Inflationary Paradigm completely. Alternative ideas to explain the problems of the Big Bang Theory include String Gas Cosmology and the Ekpyrotic scenario, though these ideas have their own limitations and flaws. This review will explore Cosmic Inflation and specifically slow-roll inflation, the type of inflation that the most successful models belong to, before briefly examining the aforementioned inflationary models.

    \newpage
    \thispagestyle{subsequentpages}
    \tableofcontents
    \newpage
\end{titlepage}

%% Contents Page and Numbering
%%-----------------------------------------------------------------------------------------------------------------------------
\pagestyle{subsequentpages}
\pagenumbering{arabic}

%% Main Body
%% -----------------------------------------------------------------------------------------------------------------------------

%% Introduction
%% -----------------------------------------------------------------------------------------------------------------------------
\section{Prologue}\label{section: Prologue}

On large scales, the universe follows the Cosmological Principle, which states that the universe is both homogenous and isotropic in every direction \cite{dimopoulosIntroductionCosmicInflation2020}. This is a rather simple idea, equating to the universe appearing to be the same everywhere on large scales; however, the reason why the universe exists in this manner is not so simple. 

Following Einstein's publication of `The Foundations of the General Theory of Relativity' \cite{Einstein1916foundationgeneraltheoryrelativity}, in 1917 Willem de Sitter proposed a model \cite{deSitter1917} parametrised by the cosmological constant from the Einstein equations. This universe would undergo rapid accelerated expansion. \cref{eq: scale factor relations}. This would later come to be known as the de Sitter model and is used to describe the expansion of Cosmic Inflation. 

The universe was first discovered to be expanding by Edwin Hubble \cite{Hubble1929relationdistanceradialvelocityextragalacticnebulae} in 1929, though non-static predictions had been derived from General Relativity \cite{Sauer2005AlbertEinsteins1916ReviewArticleGeneralRelativity}. It was first done so by Alexander Friedmann in his 1922 paper `On the Curvature of Spacetime' \cite{AlexanderFriedmannCurvatureSpace}, in which he derived what would come to be known as the Friedmann-Lema\^itre equations \cite{RomeuDerivationFriedmanequations}. In 1927, Lema\^itre independently derived the expansion and predicted the velocity-distance relation observed by Hubble \cite{Lemaitre1927}. The discovery that the universe was not static and was expanding led to research into the origins of the universe. In 1966, Stephen Hawking \cite{Hawking1966} proved a theorem that cosmic expansion implies geodesic incompleteness, referred to as the Big Bang singularity. However, despite the success of models such as these, the Big Bang did not solve all the problems regarding the origin of the universe, notably the Horizon Problem \cref{subsection: The Horizon Problem} and the Flatness Problem \cref{subsection: The Flatness Problem} still remained present.

The idea of cosmic inflation was proposed by Alan Guth in 1981 \cite{Guth1981}, outlining inflation as a potential solution to both the Horizon Problem and the Flatness Problem. The year prior, Alexander Starobinsky had proposed a gravitational model with higher order curvature corrections \cite{Starobinsky1980a}, which became one of the most successful inflation models. Since then, many different models have been proposed, including trying to embed inflationary models into theoretical frameworks such as string theory and supergravity. Inflation is strongly observationally supported, but no single inflation model has been proven correct. 



% In the years following Einstein's publication of `The Foundations of the General Theory of Relativity' \cite{Einstein1916foundationgeneraltheoryrelativity}, 2 of the simplest cosmological models were the Einstein universe and the de Sitter universe. Both were static models, though they differed in their composition, with the Einstein model containing a non-zero density for matter, for which he was required to include a cosmological constant, $\Lambda,$ to prevent the universe from collapsing in on itself, while the de Sitter model neglects ordinary matter and is dominated by this cosmological constant. The model that Einstein described is static due to his inclusion of the cosmological constant, and the de Sitter model is one that undergoes rapid expansion, and unlike the Einstein model, it satisfies the Cosmological Principle.

%% Inflation crash course
%% -----------------------------------------------------------------------------------------------------------------------------
\section{A brief crash course}\label{subsubsection: A brief crash course}
This section gives an overview of the cosmological terminology and ideas needed to understand the latter sections \cite{dimopoulosIntroductionCosmicInflation2020,baumannTASILecturesInflation2012,Narlikar1993,Ryden2016,Roos2015}. 

The Hubble Parameter, $H$, is defined as the expansion rate of the universe and is given by the formula 
\begin{equation}\label{eq: Hubble Parameter}
    H = \frac{\dot{a}}{a},
\end{equation}
where $a=a(t)$ is the scale factor of the universe and the overdots, $\dot{x}$, represent derivatives with respect to time. 
This is a key part of the 2 non-linear ordinary differential equations derived by Friedmann, with the first being 
\begin{equation}\label{eq: Friedmann Equation}
    H^2 \equiv \left( \frac{\dot{a}}{a}\right)^2 = \frac{1}{3 m_p^2}\rho - \frac{k}{a^2} \quad \Rightarrow \quad \rho = 3 m_p^2 H^2,
\end{equation}
known as the Friedmann equation where $\rho$ is the energy density, $m_p$ is the reduced Planck Mass and $k$ is the curvature, which for a flat universe $k = 0$, and is used to derive
\begin{equation}\label{eq: acceleration equation}
    \dot{H} + H^2 = \frac{\ddot{a}}{a} = - \frac{1}{6 m_p^2}\left( \rho + 3p \right),
\end{equation}
known as the acceleration equation, where $p$ is the pressure exerted by the contents of the universe. 

In an expanding universe, ($\dot{a} > 0$), filled with matter that satisfies $\rho + 3p \ge 0$, means that \cref{eq: acceleration equation} implies $\ddot{a} <0$. The indication of this condition is that there was a singularity in the finite past, $a(t\equiv0) = 0$, known as the Big Bang singularity. Combining \cref{eq: Friedmann Equation,eq: acceleration equation}, results in the continuity equation 
\begin{equation}\label{eq: Continuity equation}
    \dot{\rho} + 3H(\rho + p) = 0.
\end{equation}
Defining 
\begin{equation}\label{eq: Barotropic parameter}
    w = \frac{p}{\rho},
\end{equation}
where $w$ is the barotropic parameter for the universe, means that the continuity equation, \cref{eq: Continuity equation} may be integrated to find the relation of the energy density to the scale factor. This takes the form 
\begin{equation}\label{eq: rho to a}
    \rho \propto a^{-3(1 + w)},
\end{equation}
where $\rho$ is the total energy density of the universe. It can also be used for the individual constituent of the universe, denoted by $\rho_i$, where the energy density of each constituent is determined by its barotropic parameter, for $w=const$. The properties of each constituent and its effect on the parameters of the universe, the relations vary as shown in \cref{table: Composition relations}, where matter includes baryonic and dark matter.

If multiple constituents are contributing, then the energy density is the sum of each constituent's contribution. Combining this with the Friedmann equation, \cref{eq: Friedmann Equation}, gives the time relation of the scale factor depending on the overall barotropic parameter \cref{eq: scale factor relations}, where the scale factor relation for $w = -1$ is the relation present in the de Sitter universe model, as it is composed of only dark energy. 

\vspace{-0.25cm}
\begin{minipage}{0.5\textwidth}
    \begin{table}[H]
    \centering
    \begin{tabular}{p{3cm}||p{1cm}|p{1cm}|p{1cm}}

          $i$  & $w$ & $\rho(a)$ & $a(t)$ \\
        \hline
        \text{matter, $m$} & $0$ & $a^{-3}$ & $t^{\frac{2}{3}}$ \\

        \text{radiation, $\gamma$} & $\frac{1}{3}$ & $a^{-4}$ & $t^{\frac{1}{2}}$ \\

        \text{dark energy, $\Lambda$} & $-1$ & const & $e^{Ht}$ \\

    \end{tabular}
    \caption{The relation between each of the constituents of the universe and their relations to the universal parameters.}
    \label{table: Composition relations}
\end{table}
\end{minipage}
\hfill
    \begin{minipage}{0.5\textwidth}
        \centering
        \begin{equation}\label{eq: scale factor relations}
        a(t)\propto
        \begin{cases}
            t^{\frac{2}{3(1+w)}}, & \text{for } w \ne -1,\\[4pt]
            e^{Ht},               & \text{for } w = -1, 
        \end{cases}
    \end{equation}
\end{minipage}\\


For an accelerated rate of expansion (i.e. $\ddot{a} > 0$), $w < -\frac{1}{3}$ is required, and from \cref{table: Composition relations} it can be inferred that the dominant constituent must be dark energy. Dark energy is a substance with the interesting property of exerting a negative pressure and is responsible for the accelerated expansion of the universe. It is modelled as a fluid; however, it expands the universe by expanding the space between non-gravitationally bound objects. It is denoted by $\Lambda$ and in the currently accepted universe model, the $\Lambda$CDM model, it is dominant, causing the accelerated expansion observed. The observed composition ratio of the universe at present shows that matter and dark energy account for $\sim 31\%$ and $\sim 69\%$ of the energy density, respectively, using Planck TT,TE,EE+lowE+lensing+BAO $68\%$ limits \cite{Collaboration2020}.


%% Problems with the Hot Big Bang
%% -----------------------------------------------------------------------------------------------------------------------------
\section{Problems with the Hot Big Bang}\label{subsection: section: Problems with the Hot Big Bang}
The Big Bang refers to the initial explosion that started the universe and not to be confused with the Hot Big Bang, which instead refers to the thermal history of the universe. It describes the history of the universe, initially dominated by radiation and then by matter, but it does not include dark energy domination that occurred in very early and late times \cite{dimopoulosIntroductionCosmicInflation2020}. 

Both the density of matter and radiation are inversely proportional to the scale factor $\rho_m \propto a^{-3}, \; \rho_{\gamma} \propto a^{-4}$, meaning that the total density at the start of the universe would be very large. With all the energy and matter in a very reduced space, the temperature was incredibly high; however, as that space begins to increase, the average energy across would reduce as follows
\begin{equation}\label{eq: Temperature Relation to scale factor} 
    T \propto \frac{1}{a}.
\end{equation}

The Cosmic Microwave Background was formed during a period called decoupling, which occurred approximately 380,000 years after the Hot Big Bang. Decoupling is the period where radiation became decoupled from matter and was able to escape. Before this, radiation was trapped, continuously interacting with matter, and this first release of radiation became the CMB. 

The Hot Big Bang is supported by different forms of observational evidence, not limited to the universe expansion, the age of the universe and the CMB radiation. However, there are many problems it does not answer, such as the nature of dark energy and dark matter, as well as the origin of the structure formation. Other notable problems are the Horizon and the Flatness problem, which have to do with the universe being too uniform and are the main topic of this section.


%% The Horizon Problem
%% -----------------------------------------------------------------------------------------------------------------------------
\subsection{The Horizon Problem}\label{subsection: The Horizon Problem}

\begin{figure}[H]\label{fig: The Horizon Problem}
    \begin{center}
        \includesvg[width=0.7\textwidth]{Figures/FRWHorizon}
        \caption{On the left, the graph shows conformal time against position where $\tau_{0}$ and $\tau_{rec}$ refer to the conformal time at present and recombination (decoupling), and the 2 triangles show the areas of causally disconnected space at recombination. The right shows the same, where the 2 small circles refer to the triangles on the graph, with observers at the centre. This demonstrates the horizon problem as these 2 regions homogenous the same despite not being in causal contact \cite{baumannTASILecturesInflation2012}.}
    \end{center}
\end{figure}
\vspace{-0.5cm}
Within cosmology, a horizon refers to a boundary of causal contact that separates the observer from anything beyond it. 
The Hubble Horizon is the limit of causal connection, meaning that if an object is beyond this point, then nothing may communicate with it again as it is receding faster than the speed of light, $c$ \cite{dimopoulosIntroductionCosmicInflation2020}. The Particle Horizon is the maximum distance light could have travelled since decoupling, as it takes into account the expansion of space 
\begin{equation}\label{eq: Hubble Horizon and Particle Horizon}
    x_H(t) = \int_{t_1}^{t_2} \frac{c dt'}{a(t')}, \quad  D_H(t) = a(t)x_H(t),
\end{equation}
where $x_H(t)$ is the Hubble Horizon and $D_H(t)$ is the Particle Horizon, \cref{fig: The Flatness Problem}. 
% The Event Horizon corresponds to the maximum possible extent of causal relations where the upper limit is $\infty$, though depending on the universe model is limited to
% \begin{equation}\label{eq: Event Horizon}
%     D_H(t) = a(t)x_H(t) = a(t) \int_{0}^{\infty} \frac{c dt'}{a(t')}.
% \end{equation}

The Horizon Problem \cite{zotero-item-168} is the problem of the homogeneity that is observed in the CMB, over distances that are not causally connected as they appear to be in thermal equilibrium. The temperature seen when observing the CMB in one direction is in equilibrium with the observations taken in the opposite direction, even though it would take light at least 14 billion years to send light to each other. 

The regions were not within causal contact during the period of decoupling \cref{fig: The Horizon Problem}, meaning that interactions couldn't take place between them. This means that it is highly unlikely they would appear homogenous. The analytical description of the Horizon Problem is as follows \cite{dimopoulosIntroductionCosmicInflation2020},
\begin{equation}\label{eq: Horizon Problem}
    \frac{D_{H}(t_0)}{D_H(t)} = \frac{D_H(t_0) a(t)}{D_H(t) a(t_0)} \approx \frac{H(t) a(t)}{H(t_0) a(t_0)} = \frac{\dot{a}(t)}{\dot{a}(t_0)},
\end{equation}
where subscript $0$ refers to present time and $\dot{a} = aH$ \cref{eq: Hubble Parameter}. What this shows is that as $t \rightarrow 0$, the universe becomes increasingly causally disconnected, which raises the question of how the CMB is homogenous at distances beyond the Hubble Horizon \cref{eq: Hubble Horizon and Particle Horizon}. 



%% The Flatness Problem
%% -----------------------------------------------------------------------------------------------------------------------------
\subsection{The Flatness Problem}\label{subsection: The Flatness Problem}

The Flatness Problem is one of the topology of the universe, as the universe appears to be spatially flat despite a flat universe being unstable, meaning the initial conditions require extreme fine-tuning \cite{Lake2005}. Spacetime is dynamic and is warped by mass as stated by General Relativity \cite{Einstein1916foundationgeneraltheoryrelativity}, and so with all the mass that is present within the universe, why is it approximately a flat Euclidean plane?

The Friedmann equation, \cref{eq: Friedmann Equation}, can be recast in the form 
\begin{equation}\label{eq: Curvature Friedmann equation}
    \Omega(t) - 1 = \frac{k}{(aH)^2} \quad \text{where} \quad \Omega(t) \equiv \frac{\rho(t)}{\rho_{c}(t)}, \quad \rho_{c}(t) \equiv 3H(t)^2,
\end{equation}
where $\Omega(t)$ is the density parameter for the universe, which for a flat universe $\Omega = 1, \; k = 0$ \cite{baumannTASILecturesInflation2012}.

Using \cref{eq: Curvature Friedmann equation} \cite{dimopoulosIntroductionCosmicInflation2020},
\begin{equation}\label{eq: The Flatness Problem relation}
    \frac{|\Omega(t_0) -1 |}{|\Omega(t) -1|} = \left( \frac{a(t) H(t)}{a(t_0)H(t_0)} \right)^2 = \left( \frac{\dot{a}(t)}{\dot{a}(t_0)} \right)^2,
\end{equation}
where $|\Omega(t) - 1|$ is the deviation from flatness, \cref{fig: The Flatness Problem}. As $\dot{a}(t)$ is a decreasing function then $|\Omega(t_0) -1 | > |\Omega -1|$ for all $t > t_0$. This show that the deviation from flatness grows with time \cref{fig: The Flatness Problem}. 
This begs the question, why $\Omega(t_0)$ not either much greater or smaller than 1, as observations suggest?

\begin{figure}[H]\label{fig: The Flatness Problem}
    \begin{center}
        \includesvg[width=0.6\textwidth]{Figures/TheFlatnessProblem}
        \caption{The graph shows the density parameter against time, where the dotted line represents perfect spatial flatness. It can be seen that the expansion of the universe acts as a repeller of spatial flatness, highlighting the fine-tuning required for $\Omega =1$, as observations suggest \cite{Dimopoulos2024}.}
    \end{center}
\end{figure}
\vspace{-1cm}

%% Tensor Ratio and Spectral index problem
%% -----------------------------------------------------------------------------------------------------------------------------
\subsection{Primordial Density Perturbations}\label{subsection: Density Perturbations}

As previously stated, the cosmological principle is not exact, as demonstrated by the existence of any structure within the universe, and this cannot be explained by the Hot Big Bang. There exist density perturbations that give the CMB its characteristic look as photons become redshifted. The photons lose energy trying to overcome the gravitational potential exerted on them by the overdensity, resulting in variations in the temperature of the CMB. The differences in the temperature correlate with the density perturbations 
\begin{equation}\label{eq: Temperature to density perturbations}
    \left. \frac{\Delta T}{T} \right|_{CMB}  \simeq \left. \frac{\Delta \rho}{\rho} \right|_{CMB} \simeq 10^{-5},
\end{equation}
giving a very tiny deviation, but allows for the formation of the structure observed today. These density perturbations occurred before any gravitational growth, and have 3 key properties: they are adiabatic, nearly scale-invariant, and follow a Gaussian distribution. 

% An adiabatic perturbation is a type of perturbation where the fractional perturbation of the number density of each conserved matter type is equal to the fractional perturbation in the number density of photons. The reason that this is adiabatic is that the very early universe acted as a thermally isolated volume, and so the perturbations themselves also looked like a thermally isolated volume. This, combined with the maintenance of thermal equilibrium, means that no heat is transferred out of the system and is therefore adiabatic.  

Adiabatic perturbation occur when the fractional perturbation of the number density of each conserved matter type is equal to the fractional perturbation in the number density of photons. The very early universe acted as a thermally isolated system, meaning there is no local change in entropy per particle, for the perturbations to be adiabatic.

The perturbations are nearly scale-invariant, as the Hubble parameter is not exactly constant, so the size of the perturbations is slightly dependent on the scale when they exit the Horizon. This dependence on the scale is parametrised 
\begin{equation}\label{eq: Curvature Parameter}
    \mathcal{P}_{\zeta} \propto k^{n_s - 1} \quad \Rightarrow \quad n_s - 1 =  \dv{\ln \mathcal{P}_{\zeta}}{\ln k},
\end{equation}
where $k$ is the comoving wave number \cite{zotero-item-248} and $n_s$ is the spectral index of the curvature perturbation spectrum, $\mathcal{P}_{\zeta}$. If the perturbations were truly scale-invariant, then $n_s = 1$, however, the most recent Planck satellite observations show $n_s = 0.9649 \pm 0.0042$ \cite{Collaboration2020}. 
These perturbations have been observed to follow a Gaussian distribution to a high level of accuracy, with the search for non-Gaussianity being an active research area. 
% \begin{equation}\label{eq: Gaussian Distribution}
%     P\left(\delta_{rho}, \Delta \delta_{\rho} \right) = N \exp\left(  - \frac{\delta^2_{\rho}}{2 {< \delta^2_{\rho} >}_{\rho}}  \right)  \Delta \delta_{\rho},  \quad N_e = \frac{1}{\sqrt{2 \pi {< \delta^2_{\rho} >}_{\rho}}}
% \end{equation}

Another by-product of inflation is the generation of gravitational waves (gravitons), and the graviton spectrum is also approximately nearly scale-invariant, with an amplitude parametrised as 
\begin{equation}\label{eq: Graviton Spectrum}
    \sqrt{\mathcal{P}_{h}} = 2 \sqrt{16 \pi G} \left(\frac{H}{2 \pi}\right).
\end{equation}

These gravitational waves distort the CMB similarly to the adiabatic density perturbations, such that they can provide a consistent relation for single-field inflation. The primordial power ratio is then defined as
\begin{equation}\label{eq: Tensor ratio}
    r \equiv \frac{\mathcal{P}_{h}}{\mathcal{P}_{\zeta}} = \left( \frac{\left(\Delta T/T \right)_{grav}}{\left( \Delta T / T \right)_{\delta \rho/\rho}} \right)^2.
\end{equation}



%% Cosmic Inflation
%% -----------------------------------------------------------------------------------------------------------------------------
\section{Cosmic Inflation}\label{section: Cosmic Inflation}
Cosmic Inflation is defined as a period of superluminal expansion that occurred before the Hot Big Bang, meaning that this period of expansion in the early universe exceeded the speed of light. 

Initially, this idea seems to be impossible as it would break the laws of General Relativity, limiting anything with mass from moving faster than the speed of light. This does not apply to inflation, however, because it is not matter or energy that is being displaced with a velocity greater than light speed, but spacetime itself that is expanding. An analogy to this is if spacetime is considered to be the surface of a balloon, the balloon expands with a certain velocity, and the space on its surface grows. Considering an infinitesimal point on the surface, it has no velocity as it is not moving across the surface of the balloon, and yet it is changed from its initial absolute position, as the nature of space itself has changed \cite{dimopoulosIntroductionCosmicInflation2020}.

This period is modelled by the Inflationary Paradigm, which would solve the problems of the Hot Big Bang whilst seeding structure formation. The driving force behind this expansion is a hypothetical scalar field, with $w \approx -1$ resembling a de Sitter universe. Until 2012, fundamental scalar fields were only theorised but the discovery of the Higgs Boson changed that \cite{Collaboration2012}, strengthening the plausibility of inflationary models \cite{dimopoulosIntroductionCosmicInflation2020}. 


%% Mathematics of the Inflationary Paradigm
%% -----------------------------------------------------------------------------------------------------------------------------
\subsection{The Mathematical Description of the Inflationary Paradigm}\label{subsection: Mathematical description of Inflation}

In its simplest form the Inflationary Paradigm is modelled by a single scalar field, $\phi$, called the inflaton field. This field is minimally coupled to gravity and is parametrised in addition to the Einstein-Hilbert action 
\begin{equation}\label{eq: Einstein-Hilbert Action}
    S = \int d^4 x\, \sqrt{-g} \left( \frac{1}{2} m_p^2 R + \frac{1}{2} g^{\mu \nu} \partial_\mu \phi \partial_\nu \phi - V(\phi)  \right) = S_{EH} + S_{\phi},
\end{equation}
where $V(\phi)$ is the inflaton potential, $g = \text{det} \, g^{\mu \nu}$ where $g^{\mu \nu}$ is the spacetime metric with signature $(+---)$ and $R$ is the Ricci scalar \cite{Tong}. The action is the integral of the Lagrangian density over all space-time dimensions, $\dd[4]{x}$. The inflaton action, $S_{\phi}$, is simply an addition to the Einstein-Hilbert action, $S_{EH}$, and it describes the self-interactions within the scalar field.

A scalar field is defined as a spin-zero field, assuming a unique value at each infinitesimal point in space, $\phi \left( \boldsymbol{x},t \right)$. The inflaton field is a homogenous scalar field, $\phi = \phi(t)$, with its energy density and pressure defined as
\begin{equation}\label{eq: Density and Pressure of Homogenous scalar field}
    \rho_{\phi} = \frac{1}{2} \dot{\phi}^2 + V(\phi) \quad \text{and} \quad p_{\phi} = \frac{1}{2} \dot{\phi}^2 - V(\phi).  
\end{equation}

For quasi-de Sitter inflation, $w_{\phi} \approx -1$, the inflaton field must obey the condition $\dot{\phi}^2 \ll V$, meaning that $\phi$ varies very slowly with time.

Analogous to Newton's equations of motion for a particle, the Klein-Gordon equation describes the motion of a homogenous scalar field as 
\begin{equation}\label{eq: Klein-Gordon Equation}
    \ddot{\phi} + 3H\dot{\phi} + V' = 0, \quad \text{where} \quad V' = \derivative{V(\phi)}{\phi},
\end{equation}
where $3H$ and $V'$ correspond to the frictional and potential terms, respectively. In the case of quasi-de Sitter inflation, $\ddot{\phi}$ is negligible, resulting in 
\begin{equation}\label{eq: Slow Roll Approximation}
    3H \dot{\phi} \approx -V',    
\end{equation}
known as the slow-roll approximation. 


%% Slow Roll Inflation
%% -----------------------------------------------------------------------------------------------------------------------------
\subsection{Slow-Roll Inflation}\label{subsection: Slow-Roll Inflation}
Slow-roll inflation refers to the motion of the inflaton, $\phi$, with the potential, $V(\phi)$, slowed by the high friction term, $3H$.

%%Insert Figure showing slow-roll inflation

Inflation is measured in e-foldings, where each e-folding refers to the amount of time for the scale factor, $a(t)$, to increase by a factor of $e$ \cite{Marco2024}. The total number of e-foldings until the end of inflation is given in the 2 forms 
\begin{equation}\label{eq: No. e-foldings scale factor}
    N_e = \ln{\left(\frac{a_{end}}{a(t)}\right)} \quad \text{and} \quad \dd{N_e} = -H\dd{t}
\end{equation}
where $a_{end}$ refers to the end of inflation and $H \approx const$. The second expression can be recast into the form 
\begin{equation}\label{eq: No. e-foldings integral}
    N_e(\phi) \approx \frac{1}{m_p^2} \int_{\phi_{end}}^{\phi} \frac{V}{V'} d \phi', 
\end{equation}
using the slow-roll approximation \cref{eq: Slow Roll Approximation} and the Friedmann equation \cref{eq: Friedmann Equation} where $V \approx \rho$.

The slow-roll parameter, $\epsilon$, is the parameterisation change in the universe's expansion rate, 
\begin{equation}\label{eq: Epsilon paramter H}
    \epsilon \equiv - \frac{\dot{H}}{H^2}, \quad \Rightarrow \quad \frac{\dd{H}}{H} = \epsilon \dv{N_e}
\end{equation}
and for numerous e-foldings to occur, $\epsilon \ll 1$.

When the inflaton field dominates the energy density and satisfies $\dot{\phi}^2 \ll V(\phi)$, 
\begin{equation}\label{eq: Epsilon parameter slow roll}
    \epsilon \simeq \frac{1}{2} m_p^2 \left( \frac{V'}{V} \right)^2 \simeq \frac{3}{2}(1+w),
\end{equation}
where slow-roll inflation is possible only if $\epsilon \ll 1$ \cite{Liddle1994}.

However, the requirement for the continuation of inflation is $| \ddot{\phi} | \ll |3H \dot{\phi}|$ corresponding to the acceleration of the inflation being dominated by the frictional component of the field, as this allows for the continuation of the inflationary period. It is parametrised by the second slow-roll parameter, $\eta$, which is defined as 
\begin{equation}\label{eq: Eta slow-roll parameter}
    \eta = - m_p^2 \frac{V''}{V},
\end{equation}
which fulfils the condition, if $|\eta| \ll 1$.

$\epsilon$ and $\eta$ are the 2 parametrised conditions required of the slow-roll inflation conditions 
\begin{equation}\label{eq: Slow-roll conditions}
    \dot{\phi} \ll V(\phi) \quad \text{and} \quad |\ddot{\phi}| \ll |3H\dot{\phi}|, 
\end{equation}
which correspond to
\begin{equation}\label{eq: slow-roll parameters condition}
    \epsilon, |\eta | < 1.
\end{equation}
However, as $\phi$ is time-dependent and its magnitude decreases with time, the conditions for inflation \cref{eq: Slow-roll conditions} are eventually violated, bringing inflation to an end. This occurs at either
\begin{equation}\label{eq: End of inflation condition}
    \epsilon\left(\phi_{end}\right) = 1 \quad \text{or} \quad |\eta(\phi_{end})| = 1,
\end{equation}
so depending on the inflaton potential, either $\epsilon$ or $\eta$ will reach 1 first, ending the inflationary period, $\phi = \phi_{end}$. 

The spectral index and the tensor ratio of the density perturbations may be recast using the slow-roll parameters, which give the relations
\begin{equation}\label{eq: Spectral index using slow-roll parameters}
    n_s - 1 = \dv{\ln \mathcal{P}_{\zeta}}{\ln k} = 2 \eta - 6 \epsilon \quad \text{and} \quad
    r \equiv \frac{\mathcal{P}_{h}}{\mathcal{P}_{\zeta}} = 16 \epsilon.
\end{equation}


%% How Inflation solves the Horizon and Flatness Problem
%% -----------------------------------------------------------------------------------------------------------------------------

\subsection{How Inflation solves the Horizon and Flatness Problems}\label{subsection: How Inflation solves the Horizon and Flatness Problems}

Both the Horizon and Flatness problems are solved by inflation, as the accelerated expansion suppresses any initial anisotropy before the Hot Big Bang. Both problems are solved provided that inflation lasts $N_e \approx 60$ e-foldings. This sets a minimum time that inflation has to last, and this can be accomplished by all inflation models. This expansion creates the homogeneity in areas that are not in causal contact at the end of inflation, and so after the Hot Big Bang, they retain this homogeneity. The decay of the inflaton field at the end of inflation reforms the thermal bath that was lost due to the rapid expansion. However, in order for inflation to create the primordial density perturbations that allow for the structure formation in the universe, slow-roll inflation is required. 

\vspace{-0.15cm}
\subsubsection*{Solving Horizon Problem}\label{subsubsection: Solving the Horizon Problem}

Recalling from section \cref{subsection: The Horizon Problem}, the mathematical description of the Horizon problem is 
\begin{equation}\label{eq: Horizon Problem Short}
    \frac{D_{H}(t)}{D_H(t_{0})} = \frac{\dot{a}(t_0)}{\dot{a}(t)},
\end{equation}
which shows that in order to solve the Horizon Problem, $D_H (t)\leq D_H(t_{0})$ to allow for the regions to be causally correlated. From this, it is true that
\begin{equation}\label{eq: solving Horizon problem scale factor}
    \dot{a}(t_i) \leq \dot{a}(t_0) \quad \Rightarrow \quad \frac{\dot{a}_i}{\dot{a}_{end}} \leq \frac{\dot{a}_0}{\dot{a}_{end}},
\end{equation}
where $i$ and $end$ denote the start and end of inflation. 

After the end of inflation $\ddot{a} < 0$, so $\frac{\dot{a}_0}{\dot{a}_{end}} < 1$, therefore $\frac{\dot{a}_i}{\dot{a}_{end}}$ needs to be even smaller. As previously mentioned, inflation undergoes accelerated expansion, $\ddot{a} > 0$, which means that inflation solves the Horizon Problem under the condition that it lasts long enough. 

\vspace{-0.15cm}
\subsubsection*{Solving the Flatness Problem}\label{subsubsection: Solving the Flatness Problem}

In \cref{eq: The Flatness Problem relation} it is shown that $| \Omega(t) - 1| \propto \dot{a}^{-2}$, however, during inflation $\ddot{a} > 0$ meaning $\dot{a}$ is growing, so the deviation from flatness diminishes. This means that the criteria for inflation to solve the Flatness Problem can be written as 
\begin{equation}
    \frac{\dot{a}_i}{\dot{a}_0} = \sqrt{\frac{|\Omega_0 - 1|}{|\Omega_i - 1 |}}.
\end{equation}

From this, it can be seen that the longer inflation lasts, the smaller the deviation from flatness becomes, and so inflation solves the Flatness Problems assuming it lasts long enough. 






%% Modern Inflation Models
%% -----------------------------------------------------------------------------------------------------------------------------
\section{Models of Inflation}\label{section: Notable Models of Inflation}

\begin{figure}[H]
    \centering
    \includegraphics[width=12cm, height=6cm]{Figures/InflationaryModelsTensorRatioComparison[2208.00188].png}
    \caption{Demonstrates the tensor ratio against the spectral index for a variety of different inflationary models imposed onto the observational range. The notable models include Starobinsky Inflation ($R^2$-Inflation) and the $\alpha$-attractors as they fall within the most recent observational bounds. \cite{galloniUpdatedConstraintsAmplitude2023}}
    \label{fig: Inflationary Model Comparison}
\end{figure}

\subsection{Starobinsky Inflation}\label{subsection: Starobinsky Inflation}
Starobinsky or $R^2$ inflation was originally proposed in 1980 by Alexander Starobinsky \cite{Starobinsky1980a}, motivated by quantum corrections to gravity. It gets the name $R^2$ inflation from the fact that it is derived from a modification to the Einstein-Hilbert action for gravity \cref{eq: Einstein-Hilbert Action}. The Lagrangian density for gravity is given by
\begin{equation}\label{eq: Einstein-Hilbert Lagrangian}
    \mathcal{L} = \frac{1}{2} m_p^2 R,
\end{equation}
and Starobinsky Inflation modifies it with the addition of another $R^2$ term. It then takes the form
\begin{equation}\label{eq: Starobinsky Lagrangian}
    \mathcal{L} = \frac{1}{2} m_p^2 R + \frac{1}{2} \alpha R^2 \quad \text{where} \quad \alpha \equiv \frac{m_p^4}{16 V_0},
\end{equation}
where $V_0$ is the initial value for the inflaton potential. This Lagrangian density results in the following potential for the inflaton 
\begin{equation}\label{eq: Starobinsky Potential}
    V(\phi) = V_0 \left( 1 - e^{-\sqrt{\frac{2}{3}} \frac{\phi}{m_p}}  \right)^2.
\end{equation}

Starobinsky inflation has been one of the most successful inflation models, and its predictions lie within the observational bounds from the Planck surveys \cite{Collaboration2020} for $N_e = 60$. The results for the spectral index and tensor ratio is parametrised as 
\begin{equation}
    n_s = 1 - \frac{2}{N_e} \quad \text{and} \quad r = \frac{12}{N_e^2},
\end{equation}
in the high $N_e$ limit \cite{Kallosh2013}. 

There is potentially a tension between the Starobinsky model and the recent observations from the Atacama telescope at the time of writing \cite{Beringue2025}. The Atacama telescope measured the spectral index to $n_s = 0.9743 \pm 0.0034$ excluding the Starobinsky model at the $2\sigma$ level \cite{Bezerra-Sobrinho2025}

\subsection{\texorpdfstring{$\alpha$}{alpha}-attractors}\label{subsection: alpha-attractors}

$\alpha$-attractors are an example of a modern inflationary model family which also aligns with the observational bounds. The two main models for $\alpha$-attractors are the T-model and the E-model \cite{kalloshPresentStatusInflationary2025} which take the form:
\begin{equation}
    \text{T-model:} \quad V(\varphi) = \frac{1}{2} m^2 \varphi^2 \quad \Rightarrow \quad V(\phi) = 3\alpha m^2m_p^2 \tanh[2](\frac{1}{\sqrt{6 \alpha}} \frac{\phi}{m_p})\label{eq: T-model} \quad \text{and}
\end{equation}
\begin{equation}
    \text{E-model:} \quad V(\varphi) = \frac{\frac{1}{2} m^2 \varphi^2}{\left( 1 + \frac{1}{\sqrt{6 \alpha}} \frac{\varphi}{m_p}  \right)^2} \quad \Rightarrow \quad V(\phi) = \frac{3}{4} \alpha m^2 m_p^2 \left( 1 - e^{- \sqrt{\frac{2}{3 \alpha}} \frac{\phi}{m_p}}  \right)^2 \label{eq: E-model},
\end{equation}
with the E-model being a form of the Starobinsky potential when $V_0 = 3\alpha m^2 m_p^2$ and $\alpha = 1$.
The origin of these models lies in supergravity \cite{Kallosh2015} and are dependent on the dimensionless parameter $\alpha$. \cref{fig: Inflationary Model Comparison} shows the bounds of both models as they are parametrised by 
\begin{equation}\label{eq: Alpha-attractor n_s and r}
    n_s = 1 - \frac{2}{N_e} \quad \text{and} \quad r = \frac{12 \alpha}{N_e^2},
\end{equation}
in the high $N_e$ limit.
It demonstrates a range of $\alpha$ values, though typically in supergravity the typical values are $3 \alpha = 1,2,3,4,5,6,7$, corresponding to the choice of K\"ahler curvature \cite{Kallosh2019}.

The models are derived using the Lagrangian kinetic term for the non-canonical scalar field, $\varphi$, 
\begin{equation}\label{eq: Lagrangian kinetic term}
    \mathcal{L}_{kin} = \frac{\frac{1}{2} \partial \varphi ^2}{\left( 1 - \frac{1}{6 \alpha} \left( \frac{\varphi}{m_p} \right)^2 \right)^2},
\end{equation}
with poles at $\varphi = \pm \sqrt{6 \alpha} m_p$.  Using a canonical normalisation, $V(\varphi)$ is transformed to $V(\phi)$ giving rise to the potentials above, \cref{eq: T-model,eq: E-model}. This stretches the poles to infinity, and so this flattening of the potential allows for a plateau in the potential for slow-roll inflation. 




%% Problems with Inflation
%% -----------------------------------------------------------------------------------------------------------------------------
\section{Problems with Inflation}\label{section: Problems with Inflation}

One of the problems with cosmic inflation is that the constraints give a lot of flexibility to the parametrisation of different inflation models. This brings into question whether the Inflationary Paradigm itself is falsifiable, and specifically the Paradigm, not individual models, as they are clearly falsifiable \cite{Martin2019}. The doubt regards the entire scenario as a whole and whether there are alternative explanations for the creation of density perturbations and solutions to the Horizon and Flatness Problems. 

In parallel to the CMB, there potentially could exist a thermal background of relic gravitons, resulting from the decoupling of primordial gravitons around Planck time. The discovery of the Cosmic Graviton Background, CGB, \cite{Vagnozzi2022}, would rule out the Inflationary Paradigm as inflationary models would dilute this background, such that it would be undetectable. However, some of the alternatives to inflation allow for a CGB to remain detectable, and so the validity of inflation is dependent on the CGB being detectable. 


%% Alternatives to Inflation
%% -----------------------------------------------------------------------------------------------------------------------------

\subsection{Alternatives to Inflation}\label{section: Alternatives to Inflation}

There exist alternative propositions that attempt to solve the Horizon and Flatness Problems and to create the density perturbations observed. The main ones discussed here are String Gas Cosmology, Variable Speed of Light, the Ekpyrotic scenario, and the Matter Bounce hypothesis. While each of these propose interesting methods as alternatives to the Inflationary Paradigm, they have significantly more flaws. 

\subsubsection*{String Gas Cosmology}\label{subsubsection: String Gas Cosmology}

String Gas Cosmology, SGC, is based on the coupling of the spacetime geometry to a gas of string matter \cite{Brandenberger2008}, as an alternative explanation to inflation. As string theory naturally includes the graviton \cite{Scherk1974}, it would be able to produce the CGB if detected. An infinite temperature is prevented in this model, which causes the universe to undergo a slow contraction phase during which the universe becomes homogenous, solving the Horizon Problem. The Flatness Problem is solved as the internal dynamics of strings and branes drive the universe towards being flat. 

The problem with this proposition is that it is based entirely on speculative physics, string theory, and it produces relics such as gravitinos and magnetic monopoles \cite{Battefeld2009}. This is a major problem as it requires another phase of inflation to address these problems and to dilute them to align with observations. 


\subsubsection*{Variable Speed of Light}\label{subsubsection: Variable speed of light}

Variable Speed of Light cosmology proposes that during the early universe, the speed of light was faster than today \cite{Moffat2002}. This proposition solves the Horizon Problem as a higher speed of light means that the Hubble Horizon, or boundary of causal contact, was larger, and so the homogenous regions observed were in causal contact before the Hot Big Bang. This proposition also explains the Flatness Problem because the higher speed of light suppresses the relative effect of curvature. Primordial density perturbations are also permitted to exist within this model \cite{Moffat2016}, while being nearly scale-invariant to align with the spectral index observed. 

While this proposition solves both the Horizon and Flatness Problems while allowing for the creation of primordial density perturbations, the physics of this reduction is unclear \cite{Albrecht1999}. Some models propose $c(t)$, \cite{Barrow1999}, where it is modelled as a continuous function inversely proportional to the scale factor. However, this would require continual changes in $c$ which have not been observed. 


\subsubsection*{Ekpyrotic Scenario}\label{subsubsection: Ekpyrotic Scenario}

The Ekpyrotic model was originally introduced in an attempt to explain what occurred before the Big Bang. It was derived from String Theory, explaining the Big Bang as 2 linked branes \cite{Langlois2002}, making contact and the resulting movement apart being the expansion of the universe. It has been incorporated into cyclical cosmological models \cite{Lehners2008}, where the collapsing of the universe results in a new one. 

Initial versions of this model assumed that the universe originated as an empty BPS \cite{Kan2024} state that lasted an exponentially long time before the Hot Big Bang \cite{Khoury2001}. This assumption is what explains the Flatness Problem by just assuming flatness. More recent models however, suggest that there existed a contracting phase where $w \gg1$ that suppressed the inhomogeneities and drove the universe towards flatness. The Horizon Problem is solved as the colliding branes create global conditions for the universe. 

Problems with this model are that it is based on speculative physics and originally offered no reason for the attraction between the branes. Another problem is that the solution to the Horizon and Flatness Problems is setting the initial conditions such that they do not exist. However, consequently, this doesn't explain the fine-tuning problem and instead makes it more prominent.   


\subsubsection*{Matter Bounce}\label{subsubsection: Matter Bounce}


Matter Bounce is another alternative that involves a contracting phase followed by a bounce, during which standard Big Bang cosmology occurs. This model solves the Horizon and Flatness Problems as the contracting phase allows for the distant regions to come into causal contact as well as driving spatial flatness. One proposed mechanism behind this is that scalar fields oscillate and drive the contracting phase of the universe, and depending on the energy scale more than one bounce may occur \cite{Li2017}.

The problem with this model is the creation of ghosts due to the new scalar fields \cite{Brandenberger2011}, and the phantom behaviour of these fields is problematic except in a low-energy range. The high amount of radiation present in the very early universe is incompatible with the matter-dominated contracting phase as radiation dominates and grows faster $\rho_{\gamma} \propto a^{-4}, \rho_{m} \propto a^{-3}$. This would require very fine-tuning to be plausible which does not erase the problems of the Inflationary Paradigm.
 

%% Conclusion
%% -----------------------------------------------------------------------------------------------------------------------------

\section{Conclusion}\label{section: Conclusion}

Cosmic Inflation is a scenario constructed to solve the problems of the Hot Big Bang, specifically the Horizon Problem, the Flatness Problem while allowing for the scale-invariant density perturbations in the spectral index. The Inflationary Paradigm is the hypothesis where the rapid expansion is driven by a scalar field to create uniformity prior to the Hot Big Bang. The Horizon and Flatness Problems are solved simply by inflation lasting long enough, equating to about $N_e = 60$ for the number of e-foldings. However, slow-roll inflation produces the near scale-invariant density perturbations observed due to $H\approx const$ and the potential dominating the inflaton's kinetic term, $\dot{\phi}^2 \ll V(\phi)$. 

These requirements give a lot of flexibility to the parametrisation of the models; among them some notable models are the family of $\alpha$-attractor models and the Starobinsky inflation model. The Starobinsky model was one of the first and is one of the most successful inflationary models, being discovered within many newer models, such as the E-model from $\alpha$-attractors. Despite the success of the Inflationary Paradigm, there has been difficulty embedding it into theoretical frameworks such as string theory and supergravity.

The Inflationary Paradigm is not without its problems, most notably whether it is falsifiable itself, which has led to alternative models such as String Gas Cosmology and the Ekpyrotic scenario. These models have some promising features, but they are plagued by their own problems and specificities. Currently, the Inflationary Paradigm offers most robust and observationally supported explanation for the problems incurred by the Hot Big Bang than these alternatives do. The broad range of models means that as the observational bounds continue to be constrained, non-conforming models will be removed from contention. 


%% References
%%-----------------------------------------------------------------------------------------------------------------------------
\newpage

\printbibliography

\end{document}