\documentclass{article}


%% Package Importing
%%-----------------------------------------------------------------------------------------------------------------------------
\usepackage{outline}
\usepackage{geometry}
\geometry{a4paper, margin=1in}

\usepackage{graphicx}
\usepackage{physics}
\usepackage{titlesec}
\usepackage[backend=biber,style=numeric]{biblatex}
\addbibresource{Phys450-References.bib}
\usepackage{fancyhdr}
\usepackage{titling}
\usepackage[hidelinks]{hyperref}
\usepackage{cleveref}
\usepackage{parskip}


%% Defines the page environments
%%-----------------------------------------------------------------------------------------------------------------------------
\fancypagestyle{titlepage}{
    \setlength{\headheight}{13.6pt}
    \fancyhead[L]{Lancaster University}
    \fancyhead[R]{Department of Physics}
    \fancyfoot{}
	\renewcommand{\headrulewidth}{1pt}
	\renewcommand{\footrulewidth}{1pt} 
}

\fancypagestyle{subsequentpages}{
    \fancyfoot[C]{\thepage} 
	\renewcommand{\headrulewidth}{1pt}
	\renewcommand{\footrulewidth}{1pt}
}

%% Configures spacing for equations
%%-----------------------------------------------------------------------------------------------------------------------------
\AtBeginDocument{
  \setlength{\abovedisplayskip}{2pt}
  \setlength{\belowdisplayskip}{2pt}
  \setlength{\abovedisplayshortskip}{2pt}
  \setlength{\belowdisplayshortskip}{2pt}
}

\begin{document}

%%-----------------------------------------------------------------------------------------------------------------------------
\numberwithin{equation}{section}
\setcounter{equation}{0}
\numberwithin{equation}{subsection}
\setcounter{equation}{0}


%% Title Page
%%-----------------------------------------------------------------------------------------------------------------------------
\begin{titlepage}
    \thispagestyle{titlepage}
    \centering
    \includegraphics[width=0.5\textwidth]{Images/Physics logo.jpg}\\
    {\Huge \textbf{Literature Review: Cosmological Inflation} \par}
    \vspace{1cm}
    {\large Rhys Abercromby\par}
    \vspace{0.5cm}
    {\large Phys450: MPhys Project \par}
    \vspace{0.5cm}
    {\today \par}
    \vspace{1cm}
    {\large \textbf{Abstract}\\[0.25cm]}
    \vspace{1cm}
    \textit{    \cite{dimopoulosIntroductionCosmicInflation2020}}



\end{titlepage}

%% Contents Page and Numbering
%%-----------------------------------------------------------------------------------------------------------------------------
\pagestyle{subsequentpages}
\newpage
\pagenumbering{Roman}
\tableofcontents
\pagenumbering{arabic}
\newpage

%% Main Body
%%-----------------------------------------------------------------------------------------------------------------------------

%% Introduction
%%-----------------------------------------------------------------------------------------------------------------------------
\section{Introduction}

On large scales the universe follows The Cosmological Principle which states that the universe is both homogenous and isotropic in every direction. This is a rather simple idea, equating to the universe appearing to be the same everywhere on large scales, however, the reason as to why the universe exists this way is not so simple. 

The universe was discovered to be expanding by Edmund Hubble  \cite{baumannTASILecturesInflation2012}


\section{Cosmic Inflation}

\section{Inflationary Models}

\section{Problems with Inflation}



%% References
%%-----------------------------------------------------------------------------------------------------------------------------
\newpage

\printbibliography

\end{document}