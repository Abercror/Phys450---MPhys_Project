\documentclass[11pt]{article}

%% Package Importing
%%-----------------------------------------------------------------------------------------------------------------------------
\usepackage{outline}
\usepackage{geometry}
\geometry{a4paper, margin=1in}
\usepackage{helvet}
\usepackage{graphicx}
\usepackage{physics}
\usepackage{titlesec}
\usepackage[backend=biber,style=numeric,sorting=none]{biblatex}
\addbibresource{Phys450-References.bib}
\usepackage{fancyhdr}
\usepackage{titling}
\usepackage[hidelinks]{hyperref}
\usepackage{cleveref}
\usepackage{multicol}
\usepackage{cases}
\usepackage{float}
\usepackage{booktabs}
\usepackage{multirow}
\usepackage{bookmark}
\usepackage{footmisc}
\usepackage{caption}
\usepackage{parskip}
\usepackage[inkscapepath=images/svg-inkscape/]{svg}
\captionsetup[figure]{font=footnotesize}
\captionsetup[table]{font=footnotesize}


%% Defines the page environments
%%-----------------------------------------------------------------------------------------------------------------------------
\fancypagestyle{titlepage}{
    \setlength{\headheight}{13.6pt}
    \fancyhead[L]{Lancaster University}
    \fancyhead[R]{Department of Physics}
    \fancyfoot[C]{\thepage}
	\renewcommand{\headrulewidth}{1pt}
	\renewcommand{\footrulewidth}{1pt} 
}

\fancypagestyle{subsequentpages}{
    \fancyfoot[C]{\thepage}
	\renewcommand{\headrulewidth}{1pt}
	\renewcommand{\footrulewidth}{1pt}
}

%% Configures spacing for equations
%%-----------------------------------------------------------------------------------------------------------------------------
\AtBeginDocument{
  \setlength{\abovedisplayskip}{2pt}
  \setlength{\belowdisplayskip}{2pt}
  \setlength{\abovedisplayshortskip}{2pt}
  \setlength{\belowdisplayshortskip}{2pt}
}

%% \renewcommand lines
%%-----------------------------------------------------------------------------------------------------------------------------
\renewcommand{\arraystretch}{1.25}
\sloppy
\linespread{0.9}

%% Begin Document
%%-----------------------------------------------------------------------------------------------------------------------------
\begin{document}

%% Equation numbering
%%-----------------------------------------------------------------------------------------------------------------------------
\numberwithin{equation}{section}
\setcounter{equation}{0}
\numberwithin{equation}{subsection}
\setcounter{equation}{0}


%% Title Page
%%-----------------------------------------------------------------------------------------------------------------------------
\begin{titlepage}
    \thispagestyle{titlepage}
    \pagenumbering{roman}
    \centering
    \includesvg[width=0.7\linewidth]{Images/PhysicsLogo}\\
    \vspace{0.5cm}
    {\Huge \textbf{An overview of Cosmological Inflation} \par}
    \vspace{1cm}
    {\large Rhys Abercromby\par}∑
    \vspace{0.5cm}
    {\large Phys450: MPhys Project \par}
    \vspace{0.5cm}
    {\today \par}
    \vspace{1cm}
    \sloppy
    Despite the successes of the Big Bang Theory, there are problems that occur when examining the very earliest moments of the universe. The Inflationary Paradigm is a potential answer to these problems, however, there is a plethora of different models that attempt to explain these problems. As of now, no model has been proven, with the most successful ones being Starobinsky Inflation and $\alpha$-attractor models. These models are those that best align with the Planck 2022 observational bounds on the spectral index and tensor ratio, though there is potential for observations to disprove the Inflationary Paradigm completely. Alternative ideas to explain the problems of the Big Bang Theory, include String Gas Cosmology and Matter Bounce though these ideas have their own limitations and flaws. This review will explore Cosmic Inflation and specifically slow-roll inflation, the type of inflation that the most successful models belong to, before briefly examining the aforementioned successful inflationary models.

    \newpage
    \thispagestyle{subsequentpages}
    \tableofcontents
    \newpage
\end{titlepage}

%% Contents Page and Numbering
%%-----------------------------------------------------------------------------------------------------------------------------
\pagestyle{subsequentpages}
\pagenumbering{arabic}


%% Main Body
%% -----------------------------------------------------------------------------------------------------------------------------

%% Introduction
%% -----------------------------------------------------------------------------------------------------------------------------
\section{Prologue}\label{section: Prologue}

On large scales the universe follows The Cosmological Principle which states that the universe is both homogenous and isotropic in every direction \cite{dimopoulosIntroductionCosmicInflation2020}. This is a rather simple idea, equating to the universe appearing to be the same everywhere on large scales, however, the reason as to why the universe exists in this manner is not so simple. 

% In the years following Einstein's publication of `The Foundations of the General Theory of Relativity' \cite{Einstein1916foundationgeneraltheoryrelativity}, 2 of the simplest cosmological models where the Einstein universe and the de Sitter universe. Both were static models though they differed in their composition with the Einstein model containing a non-zero density for matter, for which he was required to include a cosmological constant, $\Lambda$ to prevent the universe from collapsing in on itself, while the de Sitter model neglects ordinary matter and is dominated by this cosmological constant. The model that Einstein described is static due to his inclusion of the cosmological constant, and the de Sitter model is one that undergoes rapid expansion, and unlike the Einstein model it satisfies The Cosmological Principle.

Following Einstein's publication of `The Foundations of the General Theory of Relativity' \cite{Einstein1916foundationgeneraltheoryrelativity}, in 1932 Willem de Sitter and Einstein proposed a model containing only dark energy \cite{Einstein1932}, parameterised as the cosmological constant included in the Einstein equations. This universe would undergo rapid accelerated expansion due to its domination by dark energy \cref{eq: scale factor relations}. This would later come to be known as the de Sitter model and is used to describe the expansion of Cosmic Inflation. 


The universe was first discovered to be expanding by Edwin Hubble \cite{Hubble1929relationdistanceradialvelocityextragalacticnebulae} in 1929, though this was a known prediction derived from General Relativity \cite{Sauer2005AlbertEinsteins1916ReviewArticleGeneralRelativity}, first done so by Alexander Friedmann in his 1922 paper `On the Curvature of Spacetime' \cite{AlexanderFriedmannCurvatureSpace}, in which he derived, what would come to be known as the Friedmann-Lema\^itre equations \cite{RomeuDerivationFriedmanequations}, as in 1927 Georges Lema\^itre independently derived them \cite{1927ASSB4749LPage49}. Friedmann demonstrated that the energy density of matter, $\rho(t)$, and the cosmological constant, $\Lambda$, were linked with the time evolution of the universe.

The discovery that the universe was not static and was expanding lead to research into the origins of the universe. A prominent theory was put forward by Stephen Hawking \cite{Hawking1966}, proposing that a space-time singularity of infinite density was a natural consequence of the Big Bang Theory 

However, despite the success of models such as these they did not solve all the problems regarding the origin of the universe, notably the Horizon Problem, \cref{subsection: The Horizon Problem}, and the Flatness Problem, \cref{subsection: The Flatness Problem}, are not solved. 

The idea of cosmic inflation was proposed by Alan Guth in 1981 \cite{Guth1981}, outlining inflation as a potential solution to both the Horizon Problem and the Flatness Problem. This was then built upon, notably by Alexander Starobinsky who independently proposed one of the most successful inflation models. Many different models have been proposed, including trying to embed inflationary models into theories such as String Theory and Supergravity, but no model has been successfully proven. 



%% Inflation crash course
%% -----------------------------------------------------------------------------------------------------------------------------
\section{A brief crash course}\label{subsubsection: A brief crash course}

The Hubble Parameter, $H$, is defined as the expansion rate of the universe and is given by the formula 
\begin{equation}\label{eq: Hubble Parameter}
    H = \frac{\dot{a}}{a}
\end{equation}
where $a$ is the scale factor of the universe. 
This is a key part of the 2 non-linear ordinary differential equations derived by Friedmann which take the form 
\begin{equation}\label{eq: Friedmann Equation}
    H^2 \equiv \left( \frac{\dot{a}}{a}\right)^2 = \frac{1}{3 m_p^2}\rho - \frac{k}{a^2} \quad \Rightarrow \quad \rho = 3 m_p^2 H^2
\end{equation}
known as the Friedmann equation for a flat universe, and is used to derive
\begin{equation}\label{eq: acceleration equation}
    \dot{H} + H^2 = \frac{\ddot{a}}{a} = - \frac{1}{6 m_p^2}\left( \rho + 3p \right) 
\end{equation}
known as the acceleration equation where the overdots, $\dot{x}$, denote derivatives with respect to time, t. 

In an expanding universe, ($\dot{a} > 0$), filled with matter that satisfies $\rho + 3p \ge 0$, means \cref{eq: acceleration equation} implies $\ddot{a} <0$. The indication of this condition is that there was a singularity in the finite past, $a(t\equiv0) = 0$ known as the Big Bang singularity. 
Combining \cref{eq: Friedmann Equation,eq: acceleration equation}, results in the continuity equation 
\begin{equation}\label{eq: Continuity equation}
    \dot{\rho} + 3H(\rho + p) = 0.
\end{equation}
Defining 
\begin{equation}\label{eq: Barotropic parameter}
    w = \frac{\rho}{p}
\end{equation}
where, $w$, is the barotropic parameter for the universe, means that the continuity equation, \cref{eq: Continuity equation} may be integrated to find the relation of the energy density, $\rho$, to the scale factor, $a$, taking the form 
\begin{equation}\label{eq: rho to a}
    \rho \propto a^{-3(1 + w)}.
\end{equation}
$\rho$ is the total energy density of the universe, though it can be used for the individual species of the universe, denoted by $\rho_x$, where the energy density of each species is determined by its barotropic parameter. Depending on what species is dominant in the universe, the relations vary as follows, with dark matter and baryonic matter grouped together: 
\begin{table}[H]\label{table: Composition relations}
    \centering
    \begin{tabular}{|p{4cm}||p{2cm}|p{2cm}|p{2cm}|}
        \hline
          $i$  & $w$ & $\rho(a)$ & $a(t)$ \\
        \hline\hline
        \text{matter, $m$} & $0$ & $a^{-3}$ & $t^{\frac{2}{3}}$ \\
        \hline
        \text{radiation, $\gamma$} & $\frac{1}{3}$ & $a^{-4}$ & $t^{\frac{1}{2}}$ \\
        \hline
        \text{dark energy, $\Lambda$} & $-1$ & $a^0$ & $e^{Ht}$ \\
        \hline
    \end{tabular}
\end{table}

If there are multiple species contributing significantly then the pressure, $p$, and energy density, $\rho$, are 
\begin{equation}\label{eq: Total rho and p}
    \rho \equiv \sum_i \rho_i \quad \quad p \equiv \sum_i p_i,
\end{equation}
where the barotropic parameter for each species is 
\begin{equation}\label{eq: barotropic parameter individual}
    w_i \equiv \frac{\rho_i}{p_i}.
\end{equation}

Combining this with the Friedmann equation, \cref{eq: Friedmann Equation}, demonstrates the relation between the scale factor and time, 
\begin{equation}\label{eq: scale factor relations}
  a(t)\propto
  \begin{cases}
    t^{\frac{2}{3(1+w)}}, & \text{for } w \ne -1,\\[4pt]
    e^{Ht},               & \text{for } w = -1
  \end{cases}
\end{equation}
where the scale factor relation for $w = -1$ is the relation present in the de Sitter universe model as it is composed of only dark energy. The Einstein universe on the other hand is composed solely of matter, $w = 0$, and so the relation of scale factor with time is $a(t) \propto t^{\frac{2}{3}}$. 

For an accelerated rate of expansion, (i.e. $\ddot{a} > 0$), requires $w < -\frac{1}{3}$ and from table \cref{table: Composition relations} it can be inferred that the dominant species must be dark energy. Dark energy is a substance with the interesting property of exerting a negative pressure, and is responsible for the expansion of the universe. It is modelled as a fluid, but it expands the universe by expanding the space between non-gravitionally bound objects. It is denoted by $\Lambda$ and in currently accepted universe model, the $\Lambda$CDM model, it is dominant, causing the accelerated expansion of the universe observed. The observed composition ratio of the universe at present from Planck \cite{Collaboration2020Planck2018resultsVICosmologicalparameters} is matter and dark energy accounting for $31\%$ and $69\%$ of the energy density respectively, using Planck TT,TE,EE+lowE+lensing+BAO $68\%$ limits.




%% Problems with the Hot big bang
%% -----------------------------------------------------------------------------------------------------------------------------

\section{Problems with the Hot Big Bang}\label{subsection: section: Problems with the Hot Big Bang}
The term the Big Bang refers to the initial explosion that started the universe and is not to be confused with the Hot Big Bang which instead refers to the thermal history of the universe that began after the Big Bang. It is the history of the universe when it is dominated by radiation and then by matter, but not the dark energy domination that occurred in very early and late times \cite{dimopoulosIntroductionCosmicInflation2020}. 

Both the density of matter and radiation are inversely proportional to time, $\rho_m \propto a^{-3}, \; \rho_{\gamma} \propto a^{-4}$, meaning that the total density at the start of the universe would be very large. With all the energy and matter in a very reduced space, the temperature was incredibly high, however, as that space begins to increase the average energy across would reduce meaning that 
\begin{equation}\label{eq: Temperature Relation to scale factor} 
    T \propto \frac{1}{a}
\end{equation} which states that the universe cools down as it expands. 

The Cosmic Microwave Background was formed during a period called decoupling which occurred approximately 1 second after the Hot Big Bang, describes the process of radiation becoming decoupled from matter and able to escape. Before this, radiation was trapped, continuously interacting with matter, but after it was released and this first release of radiation became the CMB. 

The Hot Big Bang is supported by different forms of observational evidence not limited to the universe expansion, the age of the universe and the CMB radiation. However, there are many problems it does not answer, such as the nature of dark energy and dark matter as well as the origin of the structure that is seen throughout the universe. Other notable problems are the Horizon and the Flatness problem which have to do with the universe being too uniform and are the main topic of this section.



%% The Horizon Problem
%% -----------------------------------------------------------------------------------------------------------------------------
\subsection{The Horizon Problem}\label{subsection: The Horizon Problem}
Within cosmology there are different types of horizons, the Particle Horizon, the Hubble Horizon, and the Event Horizon \cite{dimopoulosIntroductionCosmicInflation2020}. 
The Hubble Horizon is the limit of causal connection meaning that if an object is beyond this point then nothing may communicate with it again as it is receding faster than the speed of light. This distance is given by the relation
\begin{equation}\label{eq: Hubble Horizon}
    x_H(t) = \int_{t_1}^{t_2} \frac{c dt'}{a(t')},
\end{equation}
where changing the limits allows for the other horizons. 

The Particle Horizon is the maximum distance that light could have travelled since the beginning of the universe and represents the maximum boundary of the observable and unobservable universe. 
\begin{equation}\label{eq: Particle Horizon}
    D_H(t) = a(t)x_H(t) = a(t) \int_{0}^{t_0} \frac{c dt'}{a(t')}
\end{equation}
The Event Horizon corresponds to the maximum possible extent of causal relations where the upper limit is $\infty$, though depending on the universe model is limited.
\begin{equation}\label{eq: Event Horizon}
    D_H(t) = a(t)x_H(t) = a(t) \int_{0}^{\infty} \frac{c dt'}{a(t')}
\end{equation}

The Horizon problem \cite{zotero-item-168} refers to the homogeneity that is observed in the CMB, over distances that are not causally connected as they appear to be in thermal equilibrium. The temperature seen when observing the CMB in one direction is in equilibrium with the observations taken in the opposite direction, despite the fact that it would take light at least 14 billion years to send light to each other. 

The regions were not within causal contact during the period of decoupling and so the odds that all have the exact same properties is astronomically low. The analytical description of the Horizon Problem is as follows \cite{dimopoulosIntroductionCosmicInflation2020},
\begin{equation}\label{eq: Horizon Problem}
    \frac{R_{obs}(t)}{D_H(t)} = \frac{D_H(t_0) a(t)}{D_H(t) a(t_0)} \approx \frac{H(t) a(t)}{H(t_0) a(t_0)} = \frac{\dot{a}(t)}{\dot{a}(t_0)},
\end{equation}
where $R_{obs}(t)$ is the radius of the observable universe at a given time and using the relation $\dot{a} = aH$, \cref{eq: Hubble Parameter}. What this shows is that as $t \rightarrow 0$ the universe becomes increasingly causally disconnected which posits the question of how the CMB is isotropic large distances. 




%% The Flatness Problem
%% -----------------------------------------------------------------------------------------------------------------------------

\subsection{The Flatness Problem}\label{subsection: The Flatness Problem}
The Flatness Problem is one of the topology of the universe, as the universe appears to be spatially flat despite a flat universe being unstable, so the initial conditions require extreme fine-tuning \cite{Lake2005}. Spacetime is dynamic and is warped by mass as stated by General Relativity \cite{Einstein1916foundationgeneraltheoryrelativity} and so with all the mass that is present within the universe why is it approximately a flat Euclidean plane. 

Recall the Friedmann equation, \cref{eq: Friedmann Equation}, and the energy density ratio, $\Omega = \frac{\rho_i}{\rho_c}$h where $\rho_c$ is the critical density. The Friedmann equation can be rewritten in the form 
\begin{equation}\label{eq: Curvature Friedmann equation}
    \Omega(a) - 1 = \frac{k}{(aH)^2} \quad \text{where} \quad \Omega(a) \equiv \frac{\rho(a)}{\rho_{crit}(a)}, \quad \rho_{crit}(a) \equiv 3H(a)^2,
\end{equation}
where $\Omega(a)$ is defined to be time-dependent \cite{baumannTASILecturesInflation2012}.

Using the equation above \cite{dimopoulosIntroductionCosmicInflation2020},
\begin{equation}\label{eq: The Flatness Problem relation}
    \frac{|\Omega(t_0) -1 |}{|\Omega -1|} = \left( \frac{a(t) H(t)}{a(t_0)H(t_0)} \right)^2 = \left( \frac{\dot{a}(t)}{\dot{a}(t_0)} \right)^2
\end{equation}
and that $\dot{a}(t)$ is a decreasing function then $|\Omega(t_0) -1 | > |\Omega -1|$ for all $t > t_0$, showing that the deviation from flatness grows with time. 
This begs the question, why $\Omega(a_0)$, is not either much greater or smaller than 1 as observations suggest?


%% Tensor Ratio and Spectral index problem
%% -----------------------------------------------------------------------------------------------------------------------------


\subsection{Density Perturbations}\label{subsection: Density Perturbations}

As previously stated the cosmological principle is not exact, as demonstrated by the existence of any structure within the universe, and this cannot be explained by the Hot Big Bang. These perturbations from the cosmological principle are seen in the CMB as photons emitted from the CMB become redshifted when crossing regions of a higher density than the average. The photons lose energy trying to overcome the gravitational potential exerted on them by the overdensity, resulting in variations in the temperature of the CMB. The differences in the temperature correlate to the density perturbations 
\begin{equation}\label{eq: Temperature to density perturbations}
    \left. \frac{\Delta T}{T} \right|_{CMB}  \simeq \left. \frac{\Delta \rho}{\rho} \right|_{CMB} \simeq 10^{-5}
\end{equation}
giving a very tiny deviation however, this is enough to allow for the creation of the structure that exists in the universe today. 



%% Cosmic Inflation
%% -----------------------------------------------------------------------------------------------------------------------------

\section{Cosmic Inflation}\label{section: Cosmic Inflation}
Cosmic Inflation is defined as a period of superluminal inflation that occurred before the Hot Big Bang, which means that this period of expansion in the early universe exceeded the speed of light. This occurs due to accelerated expansion, $\ddot{a} > 0$, which requires the condition $w < -\frac{1}{3}$ which means that inflation was a period where dark energy was dominant, $w_{DE} = -1$. 

Initially, this idea seems to be impossible as it would break the laws of General Relativity, limits anything with mass to move slower than the speed of light. This does not apply to inflation however, because it is not matter or energy that is being displaced with a velocity greater than light speed but spacetime, itself that is expanding. An analogy to this is if spacetime is considered to be the surface of a balloon, the balloon expands with a certain velocity and the space on its surface grows. Considering an infinitesimal point on the surface, it has no velocity as it is not moving across the surface of the balloon and yet it is changed from its initial absolute position, as the nature of its space itself has changed.

% \footnote{Spacetime: the 4-dimensional fabric described by General Relativity composed of the 3-spatial dimensions and time}

As previously stated, \cref{eq: Temperature Relation to scale factor} the temperature of the universe is inversely proportional to the scale factor, so as the universe expands the temperature decreases, as the density of matter and energy is decreased. Due to the speed of inflation, the temperature of the thermal bath would be drastically depleted but at the end of inflation is when the Hot Big Bang occurs, which requires a significant thermal bath. Before explaining the mechanism of how this thermal bath is recovered, it is important to explain primordial density perturbations. 

While the Cosmological Principle states that on large scales the universe is isotropic, due to the nature of quantum mechanics small perturbations occur so on the small scale it is not isotropic. These very early perturbations can be seen today as they have been cast into the CMB, showing the inhomogeneities in the early universe. These density perturbations were the initial perturbations that occurred before any gravitational growth, and they have 3 key properties: adiabatic, scale-invariance, Gaussian. 

% An adiabatic perturbation is a type of perturbation where the fractional perturbation of the number density of each conserved matter type is equal to the fractional perturbation in the number density of photons. The reason that this is adiabatic is that the very early universe acted as a thermally isolated volume and so there perturbations themselves also looked like a thermally isolated volume. This combined with the maintaining of thermal equilibrium means that no heat is transferred out of the system and is therefore adiabatic.  

An adiabatic perturbation is a type of perturbation where the fractional perturbation of the number density of each conserved matter type is equal to the fractional perturbation in the number density of photons. The very early universe acted as a thermally isolated system meaning that despite the entropy remained constant throughout allowing for the perturbations to be adiabatic.

The perturbations are nearly scale-invariant meaning their size is not dependent on a specific scale. However, during quasi-de Sitter inflation, the Hubble parameter is not exactly constant, meaning the amplitudes of the perturbations vary slightly depending on the scale that exits the horizon at that time. This dependence is parameterised as 
\begin{equation}\label{eq: Curvature Parameter}
    \mathcal{P}_{\zeta} \propto k^{n_s - 1}    
\end{equation}
where $k$ is a constant and $n_s$ is the spectral index of the curvature perturbation, $\mathcal{P}_{\zeta}$. This deviation from scale invariance is defined as 
\begin{equation}\label{eq: Deviation from scale-invariance}
    n_s - 1 =  \dv{\ln \mathcal{P}_{\zeta}}{\ln k}
\end{equation}
and the most recent Planck satellite observations suggest that $n_s = 0.9649 \pm 0.0042$. 
These perturbations have been observed to follow a Gaussian distribution to a high level of accuracy, with the search for non-Gaussianity being an active research area. 
% \begin{equation}\label{eq: Gaussian Distribution}
%     P\left(\delta_{rho}, \Delta \delta_{\rho} \right) = N \exp\left(  - \frac{\delta^2_{\rho}}{2 {< \delta^2_{\rho} >}_{\rho}}  \right)  \Delta \delta_{\rho},  \quad N = \frac{1}{\sqrt{2 \pi {< \delta^2_{\rho} >}_{\rho}}}
% \end{equation}

Another by-product of inflation is the generation of gravitational waves (gravitons) and the graviton spectrum is also approximately scale invariant, with an amplitude parameterised as 
\begin{equation}\label{eq: Graviton Spectrum}
    \sqrt{\mathcal{P}_{h}} = 2 \sqrt{16 \pi G} \left(\frac{H}{2 \pi}\right).
\end{equation}

These gravitational waves distort the CMB similarly to the adiabatic density perturbations, such that they can provide a consistent relation for single field inflation. The ratio of the amplitude of the CMB temperature perturbations is given by 
\begin{equation}\label{eq: Tensor ratio}
    r \equiv \frac{\mathcal{P}_{h}}{\mathcal{P}_{\zeta}} = \left( \frac{\left(\Delta T/T \right)_{grav}}{\left( \Delta T / T \right)_{\frac{\delta \rho}{\rho}}} \right)^2 
\end{equation}


One way to combine this is through the Inflationary Paradigm where inflation is modelled using a scalar field. This is how the thermal bath is recovered to allow for the Hot Big Bang to occur, as at the end of inflation this scalar field decays releasing its energy which then allows for the Hot Big Bang. 



%% Mathematics of Slow roll
%% -----------------------------------------------------------------------------------------------------------------------------

\subsection{The Mathematical Description of the Inflationary Paradigm}\label{subsection: Mathematical description of Inflation}

Inflation in its simplest form is modelled by a single scalar field, $\phi$, called the inflaton. This field is coupled to gravity, and it parameterises the evolution of the inflationary energy density with respect to time. It's minimal-coupling to gravity is given as an addition to the Einstein-Hilbert action 
\begin{equation}\label{eq: Einstein-Hilbert Action}
    S = \int d^4 x\, \sqrt{-g} \left( \frac{1}{2} m_p^2 R + \frac{1}{2} g^{\mu \nu} \partial_\mu \phi \partial_\nu \phi - V(\phi)  \right) = S_{EH} + S_{\phi}
\end{equation}
where $V(\phi)$ is the inflaton potential. The action is the integral of the Lagrangian over all possible paths, hence the $d^4 x$ as that denotes the integral is occurring over the 4-dimensions of spacetime. As seen the inflaton action, $S_{\phi}$ is an addition to the Einstein-Hilbert action, and it describes the self interactions within the scalar field.

A scalar field is defined as a spin-zero field, assuming a unique value at each point in space. This means that in the case of the inflaton, $\phi = \phi(\boldsymbol{x}, t)$ where $\boldsymbol{x}$ refers to each of the spatial dimensions. This means that at each infinitesimal point in space there is a unique value for the magnitude of the scalar field which is time-dependent. A homogenous scalar field however is simply defined as $\phi = \phi(t)$ and is the same across each point in space but is still time-dependent. For this type of scalar field the density and pressure are given as 
\begin{equation}\label{eq: Density and Pressure of Homogeneous scalar field}
    \rho_{\phi} = \frac{1}{2} \dot{\phi}^2 + V(\phi)2, \quad p_{\phi} = \frac{1}{2} \dot{\phi}^2.  
\end{equation}

As previously defined for quasi-de Sitter inflation to occur $w_{\phi} \approx -1$ and this condition requires that $\frac{1}{2} \dot{\phi}^2 \ll V$, and so $\dot{\phi}$ is very small meaning $\phi$ does not vary much with time. 

Analogous to the Newton's equations of motion for a particle, the Klein-Gordon equation describes the equation of motion for a homogenous scalar field as 
\begin{equation}\label{eq: Klein-Gordon Equation}
    \ddot{\phi} + 3H\dot{\phi} + V' = 0, \quad \text{where} \quad V' = \derivative{V}{\phi}.    
\end{equation}
This describes the motion of the scalar field with a potential $V$, and a frictional term $3H$. This equation is further simplified as for quasi-de Sitter inflation $\ddot{\phi}$ is negligible which gives the resulting relation 
\begin{equation}\label{eq: Slow Roll Approximation}
    3H \dot{\phi} \approx -V'(\phi)    
\end{equation}
which is known as the slow-roll approximation. 


%% Slow Roll Inflation
%% -----------------------------------------------------------------------------------------------------------------------------
\subsection{Slow-Roll Inflation}\label{subsection: Slow-Roll Inflation}
Slow-roll inflation refers to the motion of the inflaton, $\phi$, in the potential, $V(\phi)$, as it is akin to a ball slowly rolling down a hill. 

%%Insert Figure showing slow-roll inflation

Inflation is measured in e-foldings where an e-folding refers to the amount of time for the scale factor, $a(t)$, to increase by a factor of $e$. The total number of e-foldings is given by
\begin{equation}\label{eq: No. e-foldings scale factor}
    N(\phi) = \ln{\left(\frac{a_{end}}{a(t)}\right)}
\end{equation}
where $a_{end}$ refers to the end of inflation. By substitution, the expression can be recast as
\begin{align}\label{eq: No. e-foldings integral}
    N(\phi) &= \int_{t}^{t_{end}} H dt' = \int_{\phi}^{\phi_{end}} \frac{3H}{\dot{\phi'}_{end}} d \phi'
    \\
    N(\phi) &\approx \int_{\phi_{end}}^{\phi} \frac{V}{V'} d \phi' \quad \text{using the slow roll approximation} \quad 3H\dot{\phi} \approx - V'(\phi)
\end{align}

The time-evolution of the Hubble Parameter is parameterised by the first slow-roll parameter,
\begin{equation}\label{eq: Epsilon parameter}
    \epsilon = - \frac{\dot{H}}{H^2} = - \frac{d \ln{H}}{dN}     
\end{equation}
with the $2^{nd}$ form being found using the first expression from the previous equation \cref{eq: No. e-foldings integral}, $dN = Hdt$. 

During slow-roll inflation \cite{Liddle1994}, the slow-roll parameter $\epsilon$ can be recast into the form 
\begin{equation}\label{eq: Epsilon slow-roll potential form}
    \epsilon \approx \frac{1}{2} m_p^2 \left( \frac{V'}{V} \right)   
\end{equation}
In order for accelerated expansion to occur, the condition $\epsilon \ll 1$ is required to be fulfilled, and in the case of quasi-de Sitter inflation the potential energy of the field dominates the kinetic energy of the inflaton, 
\begin{equation}\label{eq: de Sitter inflation requirement}
    \dot{\phi}^2 \ll V(\phi).
\end{equation}
To demonstrate this, $\epsilon$ can also be written in the form 
\begin{equation}\label{eq: Epsilon(w)}
    \epsilon = \frac{3}{2}(1+w),
\end{equation}
which is clearly demonstrates that $\epsilon \ll 1$ guarantees that quasi-de Sitter inflation will occur as $w > -1$. 
However, the requirement for the continuation of inflation is 
\begin{equation}\label{eq: Requirement for inflation length}
    | \ddot{\phi} | \ll |3H \dot{\phi}|,
\end{equation}
which corresponds to the acceleration of the inflation being dominated by the frictional component of the field. This is required to maintain the slow-roll of the inflation down the potential, as this increases the length of the inflation period. It is parameterised by the second slow-roll parameter, $\eta$, which is defined as 
\begin{equation}\label{eq: Eta slow-roll parameter}
    \eta = m_p^2 \frac{V''}{V}
\end{equation}
and so when $\eta \ll 1$, it requires that $|V''| < H^2$. 

$\epsilon$ and $\eta$ are the parameterisation of the 2 conditions required for slow-roll inflation \cref{eq: de Sitter inflation requirement,eq: Requirement for inflation length} to both occur and to last long enough to solve both the Horizon Problem and the Flatness Problem. For the friction to be dominant over the kinetic energy of the inflaton
\begin{equation}\label{eq: slow-roll parameters condition}
    \epsilon, |\eta | < 1,
\end{equation}
however, as $\phi$ is dependent time and is magnitude decreases with time, and so the conditions for inflation \cref{eq: de Sitter inflation requirement,eq: Requirement for inflation length} are eventually violated, bringing inflation to an end. In terms of the slow-roll parameters this is demonstrated as
\begin{equation}\label{eq: End of inflation condition}
    \epsilon\left(\phi_{end}\right) = 1, \quad |\eta(\phi_{end})| = 1. 
\end{equation}
Depending on the inflaton potential, either $\epsilon$ or $\eta$ will equate to 1 and either one doing so will end inflation, and determine the value of inflaton, $\phi_{end}$.


The spectral index and the tensor ratio of the density perturbations may be recast using the slow-roll parameters which gives the relations
\begin{align}\label{eq: Spectral index using slow-roll parameters}
    &n_s - 1 = \dv{\ln \mathcal{P}_{\zeta}}{\ln k} = 2 \eta - 6 \epsilon  \\
    &r \equiv \frac{\mathcal{P}_{h}}{\mathcal{P}_{\zeta}} = 16 \epsilon
\end{align}


%% How Inflation solves the Horizon and Flatness Problem
%% -----------------------------------------------------------------------------------------------------------------------------

\subsection{How Inflation solves the Horizon and Flatness Problems}\label{subsection: How Inflation solves the Horizon and Flatness Problems}

Both the Horizon and Flatness problems are solved by inflation as this period of superluminal expansion allows for the initial conditions of the universe to be isotropic before the Hot Big Bang. This expansion creates the homogeneity in areas that are not in causal contact at the end of inflation, and so after the Hot Big Bang they retain this homogeneity. The decay of the inflaton field at the end of inflation reforms the thermal bath that was lost due to the rapid expansion, which is possible due to the adiabatic nature of the perturbations and so the entropy decays into energy. 

\subsubsection*{Condition to solve the Horizon Problem}\label{subsubsection: Conditions to solve the Horizon Problem}

Recalling from section \cref{subsection: The Horizon Problem}, the mathematical description of the Horizon problem is 
\begin{equation}\label{eq: Horizon Problem Short}
    \frac{R_{obs}(t)}{D_H(t)} = \frac{\dot{a}(t)}{\dot{a}(t_0)}
\end{equation}
which shows that in order to solve the Horizon Problem, $R_{obs} (t_i)\leq D_H(t_0)$ to allow for the regions to be causally correlated. From this it is true that
\begin{equation}\label{eq: solving Horizon problem scale factor}
    \dot{a}(t_i) \leq \dot{a}(t_0) \quad \Rightarrow \quad \frac{\dot{a}_i}{\dot{a}_{end}} \leq \frac{\dot{a}_0}{\dot{a}_{end}}
\end{equation}
where $i$ and $end$ denote the start and end of inflation. 

After the end of inflation $\ddot{a} < 0$, so $\frac{\dot{a}_0}{\dot{a}_{end}} < 1$, therefore $\frac{\dot{a}_i}{\dot{a}_{end}}$ needs to be even smaller. As previously mentioned inflation undergoes accelerated expansion, $\ddot{a} > 0$, which means that inflation solves the Horizon Problem under the condition that it lasts long enough. 

\subsubsection*{Condition to solve the Flatness Problem}\label{subsubsection: Condition to solve the Flatness Problem}

In \cref{eq: The Flatness Problem relation} it is shown that $| \Omega(t) - 1| \propto \dot{a}^{-2}$, however, during inflation $\ddot{a} > 0$ meaning $\dot{a}$ is growing, so the deviation from flatness diminishes. This means that the criteria for inflation to solve the Flatness Problem can be written as 
\begin{equation}
    \frac{\dot{a}_i}{\dot{a}_0} = \sqrt{\frac{|\Omega_0 - 1|}{|\Omega_i - 1 |}}.
\end{equation}

From this it can be seen that the longer inflation lasts the smaller the deviation from flatness becomes and so inflation solves the Flatness Problems assuming it lasts long enough. 


Both the Horizon and Flatness Problems are solved by inflation provided that inflation lasts long enough. This time period for inflation to solve them is about 60 e-foldings. This sets a minimum time that inflation has to last and this can be accomplished by all inflation models. However, in order for inflation to create the primordial density perturbations that allows for the structure formation in the universe, only specific inflation models can cause that. The Inflationary Paradigm where dark energy is modelled as a scalar field allows for slow-roll inflation to occur, and it is slow-roll inflation that allows for the deviation of the spectral index to align with the observations that are seen today. 




%% Modern Inflation Models
%% -----------------------------------------------------------------------------------------------------------------------------
\section{Models of Inflation}\label{section: Notable Models of Inflation}

\begin{figure}[H]
    \centering
    \includegraphics[width=12cm, height=6cm]{Figures/InflationaryModelsTensorRatioComparison[2208.00188].png}
    \caption{Demonstrates the tensor ratio against the spectral index for a variety of different inflationary models imposed onto the observational range. The notable models include Starobinsky Inflation ($R^2$-Inflation) and the $\alpha$-attractors as they fall within the most recent observational bounds. \cite{galloniUpdatedConstraintsAmplitude2023}}
    \label{fig: Inflationary Model Comparison}
\end{figure}

\subsection{Starobinsky Inflation}\label{subsection: Starobinsky Inflation}
Starobinsky or $R^2$ inflation was originally proposed in 1980 by Alexander Starobinsky \cite{Starobinsky1980a} and in doing so coined the name cosmic inflation. It gets the name $R^2$ inflation from the fact that it is derived from a modification to the Einstein-Hilbert action for gravity, \cref{eq: Einstein-Hilbert Action}, where $R$ is the Ricci scalar. The Lagrangian density for gravity is given by
\begin{equation}\label{eq: Einstein-Hilbert Lagrangian}
    \mathcal{L} = \frac{1}{2} m_p^2 R
\end{equation}
and Starobinsky Inflation modifies it with the addition of another $R^2$ term. It then takes the form
\begin{equation}\label{eq: Starobinsky Lagrangian}
    \mathcal{L} = \frac{1}{2} m_p^2 R + \frac{1}{2} \alpha R^2 \quad \text{where} \quad \alpha \equiv \frac{m_p^4}{16 V_0}
\end{equation}
where $V_0$ is the initial value for the inflaton potential. This Lagrangian density results in the following potential for the inflaton 
\begin{equation}\label{eq: Starobinsky Potential}
    V(\phi) = V_0 \left( 1 - e^{\sqrt{\frac{2}{3}} \frac{\phi}{m_p}}  \right)^2.
\end{equation}

Starobinsky inflation has been one of the most success inflation models and its predictions like within the observational bounds from Planck for $N = 60$. There is currently a tension with the Starobinsky model due to the recent observations from the Atacama telescope, though these were the first publications of results and so may change in the future. 

\subsection{\texorpdfstring{$\alpha$}{alpha}-attractors}\label{subsection: alpha-attractors}
$\alpha$-attractors are a modern inflation model that have been proposed, which also align with the observational bounds. The two main models for $\alpha$-attractors are the T-model and the E-model \cite{kalloshPresentStatusInflationary2025} which take the form:
\begin{align}\label{eq: alpha-attractor models}
    \text{T-model:} \quad V(\varphi) = \frac{1}{2} m^2 \varphi^2 \quad & \Rightarrow \quad V(\phi) = 3\alpha m^2m_p^2 \tanh[2](\frac{1}{\sqrt{6 \alpha}} \frac{\phi}{m_p^2}) \quad \text{and}\\
    \text{E-model:} \quad V(\varphi) = \frac{\frac{1}{2} m^2 \varphi^2}{\left( 1 + \frac{1}{\sqrt{6 \alpha}} \frac{\varphi}{m_p}  \right)^2} \quad & \Rightarrow \quad V(\phi) = \frac{3}{4} \alpha m^2 m_p^2 \left( 1 - e^{- \sqrt{\frac{2}{3 \alpha}} \frac{\phi}{m_p}}  \right)^2
\end{align}
with the E-model being a form of the Starobinsky potential when $V_0 = 3\alpha m^2 m_p^2$ and $\alpha = 1$.
The origin of these models lies in supergravity \cite{Kallosh2015} and are dependent on the dimensionless parameter $\alpha$. Figure \cref{fig: Inflationary Model Comparison} shows the bounds of both the models as they have the same tensor ratio to spectral index. It demonstrates a range of $\alpha$ values though typically in supergravity the typical values are $3 \alpha = 1,2,3,4,5,6,7$ \cite{Kallosh2019}. 

The models are derived using the Lagrangian kinetic term for the non-canonical scalar field, $\varphi$, 
\begin{equation}\label{eq: Lagrangian kinetic term}
    \mathcal{L}_{kin} = \frac{\frac{1}{2} \partial \varphi ^2}{\left( 1 - \frac{1}{\sqrt{6 \alpha}} \left( \frac{\varphi}{m_p} \right)^2 \right)^2}
\end{equation}
with poles at $\varphi = \pm \sqrt{6 \alpha} m_p$.  Using a canonical normalisation, $V(\varphi)$ is transformed to $V(\phi)$ giving rise to the potentials above, Eq.\eqref{eq: alpha-attractor models}. This stretches the poles to infinity and so this flattening of the potential allows for a plateau in the potential to give the required behaviour to the inflaton.




%% Problems with Inflation
%% -----------------------------------------------------------------------------------------------------------------------------
\section{Problems with Inflation}\label{section: Problems with Inflation}

One of the major problems with cosmic inflation is that there is a very large amount of flexibility in the predictions of individual inflationary models. This brings into question whether the Inflationary Paradigm itself is falsifiable, and specifically the Paradigm not individual models as they are clearly falsifiable. The doubt regards the entire scenario as a whole and whether there are alternative explanations for the creation of density perturbations and solutions to the Horizon and Flatness Problems. 

In parallel to the CMB, there potentially could exist a thermal background of relic gravitons, resulting from the decoupling of primordial gravitons around Planck time. The discovery of the CGB \cite{Vagnozzi2022}, Cosmic Graviton Background, would rule out the Inflationary Paradigm as inflationary models would wash this out meaning that it would be undetectable. However, some of the alternatives to inflation allow for a CGB to remain detectable and so the validity of inflation is dependent the CGB being detectable. 


%% Alternatives to Inflation
%% -----------------------------------------------------------------------------------------------------------------------------

\subsection{Alternatives to Inflation}\label{section: Alternatives to Inflation}

\subsubsection*{String Gas Cosmology}\label{subsubsection: String Gas Cosmology}

String Gas Cosmology, SGC, uses degrees of freedom and symmetries present in string theory, and is based on the coupling of the background space-time geometry to a gas of closed string matter \cite{Brandenberger2008}. String Theory naturally includes the graviton \cite{Scherk1974} hence allowing for the potential emergence of the CGB, and like in string theory, SGC compacts all spacial dimensions and assumes that they are toroidal with a radius for the torus being $R$. 

Strings have 3 types of modes that they may exist in: momentum modes representing the energy of the strings, oscillatory modes which represent the fluctuations and the winding modes which refer to the number of times the string is wrapped around the torus. The number of the oscillatory states increases exponentially with an increase in energy however there is a limiting temperature for a string gas to be in thermal equilibrium and so the temperature remains constant if the volume changes adiabatically within the Hagedorn regime. 

One of the major problems of string gas cosmology is that stabilising all the string moduli, as the size and shape of the extra dimensions must also be stabilised. SGC was originally developed using tools from a specific type of string theory, heterotic string theory, and while it does not require heterotic strings they are most commonly used, and they support the T-duality symmetry, which is a symmetry through the compactification of the radius $R \rightarrow 1/R$, which transforms one string theory to another meaning that they are dualities of each other under this symmetry \cite{Brandenberger2011}. 

\subsubsection*{Variable Speed of Light}\label{subsubsection: Variable speed of light}

Variable Speed of Light cosmology, proposes that during the early universe the speed of light was faster than today \cite{Moffat2002}. This proposition solves the Horizon Problem as a higher speed of light means that the Hubble Horizon, or boundary of causal contact, was larger, and so the homogenous regions observed were in causal contact before the Hot Big Bang. This proposition also explains the Flatness Problem by asserting that the higher speed of light stretches the universe and then the speed of light drastically decreases, flattening the curvature to what is observed. Primordial density perturbations are also permitted to exist within this model \cite{Moffat2016}, while being nearly-scale invariant to align with the spectral index observed. 

While this proposition solves both the Horizon and Flatness Problems while allowing for the creation of primordial density perturbations, the physics of this reduction is unclear \cite{Albrecht1999}. Some models propose $c(t)$, \cite{Barrow1999} where it is modelled as a continuous function inversely proportional to the scale factor. However, this would require continual changes in $c$ has not been observed. 


\subsubsection*{Ekpyrotic Scenario}\label{subsubsection: Ekpyrotic Scenario}



\subsubsection*{Matter Bounce}\label{subsubsection: Matter Bounce}

Matter bounce is another alternative to the Inflationary Paradigm that obtains the scale-invariant density perturbations, that exit the Hubble radius in a contracting matter-dominated phase that occurs before the Big Bang. After the contracting phase a bounce follows and then the standard phases of Big Bang cosmology occur. This approach solves the Horizon and the Flatness Problems similarly to the Inflationary Paradigm, but it also avoids reaching a singularity at the time of the Big Bang \cite{Li2017}. 

The typical way this idea is constructed is through the existence of a new scalar field, and so with a canonical Lagrangian, oscillations from the scalar field can drive the contracting phase when the ratio of the pressure to the energy density averages 0. With an increasing energy scale of the universe, extra terms appear in the Lagrangian that drive a nonsingular bounce. 

This alternative to the Inflationary Paradigm also has its problems with one of the big ones being the creation of ghosts that occur form the creation of the different matter fields \cite{Brandenberger2011}. The phantom behaviour of these fields is not problematic so long as the energy density remains smaller than those corresponding to the phantom field mass scale. Under these conditions it may be considered to be a low-energy effective field theory, however, a scenario involving this is unstable with the presence of radiation which was every present in the very early universe. 
 

%% Conclusion
%% -----------------------------------------------------------------------------------------------------------------------------

\section{Conclusion}\label{section: Conclusion}

Cosmic Inflation is a scenario constructed to solve the problems of the Hot Big Bang, specifically the Horizon Problem, the Flatness Problem and allows for the scale-invariant density perturbations in the spectral index. The Inflationary Paradigm, which refers to a period of quasi-de Sitter inflation dominated by dark energy, being modelled as a scalar field with a sufficiently strong frictional term producing "slow-roll inflation" solves the aforementioned problems. The solution to the Horizon and Flatness Problems is solved simply by inflation lasting long enough equating to about $N = 60$ for the number of e-foldings, however, slow-roll inflation is required to give the observational relation between the tensor ratio and the spectral index. 

These requirements allow for a very high amount of flexibility between different models, however, out of the many that exist 2 of the most notable are the $\alpha$-attractor models and the Starobinsky inflation model. The Starobinsky model is both one of the oldest to exist, and one of the most successful inflationary with it being found within many newer models such as the E-model from $\alpha$-attractors. 

Despite the success of the Inflationary Paradigm, it is still plagued by problems most notably whether it is falsifiable itself, which has lead to alternative models such as String Gas Cosmology and Matter Bounce. These models have some promising features, but they are plagued by their own problems and specificities. Currently, inflation offers the best explanation to the problems incurred by the Hot Big Bang than these alternatives do and the broad range of models means that as the observational bounds continue to be limited, non-conforming models may be removed from contention. 


%% References
%%-----------------------------------------------------------------------------------------------------------------------------
\newpage

\printbibliography

\end{document}